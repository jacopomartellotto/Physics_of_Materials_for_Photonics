\documentclass[11pt,a4paper]{book}

% -----------------------
% Pacchetti base
% -----------------------
\usepackage[utf8]{inputenc}
\usepackage[T1]{fontenc}
\usepackage[italian]{babel}
\usepackage{amsmath,amssymb}
\usepackage{graphicx}
\usepackage{hyperref}
\usepackage{geometry}
\usepackage{fancyhdr}
\usepackage{titlesec}
\usepackage{parskip}
\usepackage{color}
\usepackage{siunitx}
\usepackage{physics}

% -----------------------
% Layout
% -----------------------
\geometry{margin=2.5cm}
\pagestyle{fancy}
\fancyhf{}
\fancyfoot[C]{\thepage}
\fancyhead[L]{\leftmark}
\renewcommand{\headrulewidth}{0.4pt}

% -----------------------
% Colori e link
% -----------------------
\hypersetup{
    colorlinks=true,
    linkcolor=blue,
    citecolor=blue,
    urlcolor=blue,
    pdftitle={Appunti di Fisica dei Materiali per la Fotonica}
}

% -----------------------
% Stile titoli capitoli
% -----------------------
\titleformat{\chapter}[display]
  {\normalfont\huge\bfseries}
  {\filleft\Huge\thechapter}
  {1ex}
  {\titlerule\vspace{1ex}\filright}
  [\vspace{1ex}\titlerule]

% -----------------------
% Copertina
% -----------------------
\begin{document}
\begin{titlepage}
    \centering
    \vspace*{2cm}
    
    {\scshape\LARGE Università degli Studi di NomeUniversità \par}
    \vspace{1.5cm}
    {\Huge\bfseries Appunti di\\[0.5cm]
    Fisica dei Materiali per la Fotonica\par}
    \vspace{2cm}
    {\Large\itshape Corso tenuto dal Prof. Nome Cognome\par}
    \vfill
    {\large Appunti a cura di Nome Cognome Studente\par}
    \vspace{0.5cm}
    {\large \today\par}
\end{titlepage}

% -----------------------
% Indice
% -----------------------
\tableofcontents
\newpage

% -----------------------
% Capitoli inclusi
% -----------------------
\chapter{Introduzione alla Fotonica}

\section{Cos'è la fotonica}
La fotonica è la scienza che studia la generazione, il controllo, la propagazione e la rilevazione della luce, ovvero dei fotoni.

Come spiegato in \cite{saleh1991fotonica}, la propagazione elettromagnetica nei materiali è alla base della fotonica moderna.

\section{Applicazioni principali}
\begin{itemize}
    \item Comunicazioni ottiche
    \item Sensori
    \item Laser a semiconduttore
    \item Ottica integrata
\end{itemize}

Approfondimenti su strutture fotoniche avanzate si trovano in \cite{joannopoulos2008molding}.

% \input{chapters/capitolo2.tex}
% aggiungi altri capitoli come vuoi

% -----------------------
% Bibliografia (opzionale)
% -----------------------
% \newpage
% \bibliographystyle{plain}
% \bibliography{bibliography}

\end{document}
